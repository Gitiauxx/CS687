\documentclass{article}

\usepackage{mathtools}

\DeclarePairedDelimiter\abs{\lvert}{\rvert}%
\DeclarePairedDelimiter\norm{\lVert}{\rVert}%
\makeatletter
\let\oldabs\abs
\def\abs{\@ifstar{\oldabs}{\oldabs*}}

\usepackage[utf8]{inputenc}
\usepackage{pgfplots}

\pgfplotsset{
    discard if not/.style 2 args={
        x filter/.code={
            \edef\tempa{\thisrow{#1}}
            \edef\tempb{#2}
            \ifx\tempa\tempb
            \else
                \def\pgfmathresult{inf}
            \fi
        }
    }
}

\usepackage{subcaption}
\usepackage{gensymb}
\usepackage{amsmath,amsfonts,amssymb,amsthm,epsfig,epstopdf,titling,url,array}
\usepackage{enumerate}
\usepackage[a4paper, total={6in, 8in}]{geometry}

\usepackage{algorithm}
\usepackage[noend]{algpseudocode}

\makeatletter
\def\BState{\State\hskip-\ALG@thistlm}
\makeatother

\newcommand{\mathbbm}[1]{\text{\usefont{U}{bbm}{m}{n}#1}}
\newtheorem{thm}{Theorem}[section]
\newtheorem*{thmt*}{Theorem}
\newtheorem{lem}[thm]{Lemma}
\newtheorem{assumption}{Assumption}
\newtheorem{prop}[thm]{Proposition}
\pgfplotsset{compat=1.15}

\newtheorem{defn}{Definition}[section]
\title{Homework 1 -- Xavier Gitiaux}
\author{}
\date{February 2019}

\begin{document}
\maketitle

\section{Exercise 1}
\paragraph{Joint Distribution}
Since $B$ and $C$ are independent given $A$, $P(B, C|A)= P(B|A)P(C|A)$ so by Baye's rule,
\begin{equation}
\nonumber
Pr(A, B, C)= P(A)P(B, C|A) = P(A)P(B|A)P(C|A).
\end{equation}
Therefore 

$$P(A=a, B=b, C=c) = (0.2)(0.9)(0.7)= 0.126$$
$$P(A=a, B=\neg b, C=c) = (0.2)(0.1)(0.7)= 0.014
$$
$$P(A=a, B=\neg b, C=\neg c) = (0.2)(0.1)(0.3)=0.006$$
$$P(A=a, B= b, C=\neg c) = (0.2)(0.9)(0.3)=0.054$$
$$P(A=\neg a, B=b, C=c) = (0.8)(0.4)(0.5)=0.16$$
$$P(A=\neg a, B=\neg b, C=c) = (0.8)(0.6)(0.5)=0.24
$$
$$P(A=\neg a, B=\neg b, C=\neg c) = (0.8)(0.6)(0.5)=0.24$$
$$P(A=\neg a, B= b, C=\neg c) = (0.8)(0.4)(0.5)=0.16$$

Then $P(B, C)=P(B,C, A=a)P(A=a) + P(B, C, A=\neg a)P(A=\neg a)= P(B,C, A=a)(0.2) + P(B,C, A=\neg a)(0.8)$. 
So 
$$P(B=b, C=c)= 0.1532$$
$$P(B=b, C=\neg c)= (0.054)(0.2) + (0.16)(0.8) = 0.1388$$
$$P(B=\neg b, C=\neg c)= (0.006)(0.2) + (0.24)(0.8) = 0.1932$$
$$P(B=\neg b, C=c)= (0.014)(0.2) + (0.24)(0.8) = 0.1948
$$

\bigskip
Using Bayes' rule, 
\begin{equation}
\nonumber
P(A|B=b)=\frac{P(A,B=b)}{P(B=b)}=\frac{P(B=b|A)}{P(B=b)}P(A)
\end{equation}
Moreover, $P(B=b)= P(A=a)P(B=b|A=a) + P(A=\neg a)P(B|A=\neg a)=(0.2)(0.9) + (0.8)(0.4) = 0.5$
So $$P(A=a|B=b)=\frac{0.9}{0.5}(0.2)=0.36.$$
And $$P(A=\neg a|B=b)=0.64.$$

\bigskip
Similarly, 
\begin{equation}
\nonumber
P(A|C=c)=\frac{P(A,C=c)}{P(C=c)}=\frac{P(C=c|A)}{P(C=c)}P(A)=
\end{equation}
Moreover, $P(C=c)= P(A=a)P(C=c|A=a) + P(A=\neg a)P(C=c|A=\neg a)=(0.2)(0.7) + (0.5)(0.4) = 0.34$
So $$P(A=a|B=b)=\frac{0.7}{0.34}(0.2)=7/17$$ and 
$$P(A=\neg a|B=b)=10/17$$

\bigskip
Lastly, 
\begin{equation}
\nonumber
P(A=a|B=b, C=c)=\frac{P(A=a,C=c, B=b)}{P(B=b, C=c)}=0.126/0.1532
\end{equation}

\bigskip
We have 
\begin{equation}
P(A|B=b, C=c) = \frac{P(B=b, C=c |A)}{P(B=b, C=c)}P(A)=\frac{P(B=b|A)P(C=c|A)}{P(B=b, C=c)}P(A),
\end{equation}
the  last equation holding because $B$ and $C$ are independent given $A$. 

\section{Exercise 2}

First, 

\begin{equation}
P(x,y|e)= \frac{P(x,y,e)}{P(e)}= \frac{P(x|y, e)P(y, e)}{P(e)}= P(x|y, e)P(y|e).
\end{equation}

Secondly, 
\begin{equation}
P(y|x, e)= \frac{P(x,y,e)}{P(x, e)}= \frac{P(x|y, e)P(y, e)}{P(x,e)}= P(x|y, e)\frac{P(y|e)}{P(x, e)}.
\end{equation}

Lastly,
\begin{equation}
\frac{P(a,b,c}{P(b,c)}=P(a|b,c)=P(b|a, c)= \frac{P(a,b,c}{P(a,c)}.
\end{equation}
Therefore, if $P(a|b,c)=P(b|a, c)$, then $P(b,c)=P(a,c)$, which implies that
\begin{equation}
P(b|c)=\frac{P(b,c)}{P(c)}=\frac{P(a,c)}{P(c)}= P(a|c).
\end{equation}

\section{Exercise 3}
\subsection{Question (a)}
$P(toothache)=0.108 + 0.012 + 0.016 + 0.064=0.2$

\subsection{Question (b)}
$P(cavity)=0.108 + 0.012 + 0.072 + 0.008=0.2$

\subsection{Question (c)}
$P(toothache | cavity)=\frac{0.108 + 0.012}{0.2}=0.6$

\subsection{Question (d)}
$$P(cavity|toothache \lor catch)=\frac{P((cavity \land toothache) \lor ((cavity \land catch))}{P(toothache \lor catch)}$$

Moreover, $P(toothache \lor catch) = P(toothache) + P(catch) - P(toothache \land catch)= 0.2 + (0.108 + 0.016 + 0.072 + 0.144) - 0.108= 0.432$. And 

$$P(cavity \land (toothache \lor catch))= 0.108 + 0.012 + 0.072 = 0.192$$.

Therefore, 
$$P(cavity|toothache \lor catch)= \frac{0.192}{0.432}=\frac{4}{9}.$$


\section{Exercise 4}
\subsection{Question (a)}
$P(movie)=0.5=P(song)$. 

\subsection{Question (b)}
$$P(Perfect|movie)= \frac{2}{8}=\frac{1}{4}$$
$$P(Perfect|song)= \frac{1}{8}$$
$$P(Storm|movie)= \frac{0}{8}=0$$
$$P(Storm|song)= \frac{1}{8}$$

\subsection{Question (c)}
With Laplace smoothing,
\begin{equation}
P(Perfect \;  Storm| movie)= \frac{1}{8 + 12}= \frac{1}{20}
\end{equation}
\begin{equation}
P(Perfect \;  Storm| song)= \frac{1}{8 + 12}= \frac{1}{20}
\end{equation}
Therefore, $P(Perfect \; Storm) = \frac{1}{20}$ and 

$$P(movie|Perfect \; Storm) = \frac{P(Perfect \;  Storm| movie)}{P(Perfect \; Storm)}P(movie)= 0.5$$

It makes sense: since ``Perfect Storm" does not occur in movie nor in song, we assign a probability $0.5$ to the events $(movie|Perfect \; Storm)$ and $(song|Perfect \; Storm)$

\subsection{Question (d)}
Without Laplacian smoothing, since $P(Perfect \; Storm)=0$, $P(movie|Perfect \; Storm)$ is indefinite. 



\end{document}